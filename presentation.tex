\documentclass[aspectratio=169,dvipdfmx]{beamer}

% テーマ設定
\usetheme{Madrid}
\usecolortheme{default}
\usefonttheme{professionalfonts}

% 日本語設定
\usepackage{luatexja}
\usepackage{luatexja-fontspec}
\setmainjfont{IPAexGothic}
\setsansjfont{IPAexGothic}

% パッケージ
\usepackage{graphicx}
\usepackage{booktabs}
\usepackage{multirow}
\usepackage{amsmath}
\usepackage{hyperref}

% タイトル情報
\title{オーストラリア天気予測}
\subtitle{明日の雨予測モデルの構築と比較}
\author{}
\date{\today}
\institute{}

% カスタマイズ
\setbeamertemplate{navigation symbols}{}
\setbeamertemplate{footline}[frame number]

\begin{document}

%============================================
% スライド1: タイトル
%============================================
\begin{frame}
    \titlepage
\end{frame}

%============================================
% スライド2: 目次
%============================================
\begin{frame}{目次}
    \tableofcontents
\end{frame}

%============================================
% セクション1: 課題概要
%============================================
\section{課題概要}

%============================================
% スライド3: 課題の背景
%============================================
\begin{frame}{課題の背景}
    \begin{block}{目的}
        オーストラリアの気象データを用いて、\textbf{明日が雨になるか否か}を予測する2値分類モデルを構築する
    \end{block}

    \vspace{0.5cm}

    \begin{itemize}
        \item \textbf{データソース}: Kaggle - Rain in Australia
        \item \textbf{期間}: 2007年11月〜2017年6月(約10年分)
        \item \textbf{観測地点}: オーストラリア全土49箇所
        \item \textbf{応用}: 農業計画、イベント運営、交通管理など
    \end{itemize}
\end{frame}

%============================================
% スライド4: データセット概要
%============================================
\begin{frame}{データセット概要}
    \begin{columns}[T]
        \begin{column}{0.5\textwidth}
            \textbf{基本情報}
            \begin{itemize}
                \item レコード数: 145,460件
                \item 特徴量数: 23カラム
                \item ターゲット: RainTomorrow (Yes/No)
            \end{itemize}

            \vspace{0.3cm}

            \textbf{クラス分布(不均衡)}
            \begin{itemize}
                \item No(雨なし): 78\%
                \item Yes(雨あり): 22\%
            \end{itemize}
        \end{column}

        \begin{column}{0.5\textwidth}
            \begin{figure}
                \centering
                \includegraphics[width=\textwidth]{figures/target_distribution.png}
                \caption{ターゲット変数の分布}
            \end{figure}
        \end{column}
    \end{columns}
\end{frame}

%============================================
% セクション2: データ分析
%============================================
\section{データ分析(EDA)}

%============================================
% スライド5: 特徴量の種類
%============================================
\begin{frame}{特徴量の種類}
    \begin{columns}[T]
        \begin{column}{0.5\textwidth}
            \textbf{数値変数(16個)}
            \begin{itemize}
                \item 気温: MinTemp, MaxTemp, Temp9am, Temp3pm
                \item 降水: Rainfall, Evaporation
                \item 日照: Sunshine
                \item 風速: WindGustSpeed, WindSpeed9am/3pm
                \item 湿度: Humidity9am/3pm
                \item 気圧: Pressure9am/3pm
                \item 雲量: Cloud9am/3pm
            \end{itemize}
        \end{column}

        \begin{column}{0.5\textwidth}
            \textbf{カテゴリ変数(6個)}
            \begin{itemize}
                \item Location: 49観測地点
                \item WindGustDir: 突風方向(16方位)
                \item WindDir9am: 9時の風向
                \item WindDir3pm: 15時の風向
                \item RainToday: 本日の雨(Yes/No)
                \item \textbf{RainTomorrow}: ターゲット
            \end{itemize}
        \end{column}
    \end{columns}
\end{frame}

%============================================
% スライド6: 欠損値の分析
%============================================
\begin{frame}{欠損値の分析}
    \begin{columns}[T]
        \begin{column}{0.45\textwidth}
            \textbf{高欠損率の変数}
            \begin{itemize}
                \item Sunshine: 48.0\%
                \item Evaporation: 43.2\%
                \item Cloud3pm: 40.8\%
                \item Cloud9am: 38.4\%
            \end{itemize}

            \vspace{0.3cm}

            \textbf{対処方針}
            \begin{itemize}
                \item Location別の中央値で補完
                \item カテゴリ変数は最頻値で補完
            \end{itemize}
        \end{column}

        \begin{column}{0.55\textwidth}
            \begin{figure}
                \centering
                \includegraphics[width=\textwidth]{figures/missing_values.png}
                \caption{欠損値ヒートマップ}
            \end{figure}
        \end{column}
    \end{columns}
\end{frame}

%============================================
% セクション3: 前処理
%============================================
\section{前処理パイプライン}

%============================================
% スライド7: データ分割
%============================================
\begin{frame}{データ分割(時系列)}
    \begin{block}{時系列分割の重要性}
        未来のデータで過去を予測することを防ぐため、\textbf{時間順序を保持}してデータを分割
    \end{block}

    \vspace{0.5cm}

    \begin{table}[h]
        \centering
        \begin{tabular}{lccc}
            \toprule
            データセット & 期間 & レコード数 & 割合 \\
            \midrule
            Train & 〜2015/06 & 約113,000 & 80\% \\
            Validation & 2015/07〜2016/06 & 約14,000 & 10\% \\
            Test & 2016/07〜 & 約14,000 & 10\% \\
            \bottomrule
        \end{tabular}
    \end{table}
\end{frame}

%============================================
% スライド8: 前処理の詳細
%============================================
\begin{frame}{前処理の詳細}
    \textbf{1. 欠損値処理}
    \begin{itemize}
        \item 数値変数: Location別の中央値で補完
        \item カテゴリ変数: Location別の最頻値で補完
        \item RainTomorrow欠損行は削除
    \end{itemize}

    \vspace{0.3cm}

    \textbf{2. 特徴量エンコーディング}
    \begin{itemize}
        \item 風向(16方位): サイクリカルエンコーディング($\sin\theta$, $\cos\theta$)
        \item RainToday: バイナリ(Yes=1, No=0)
        \item Location: ラベルエンコーディング
    \end{itemize}

    \vspace{0.3cm}

    \textbf{3. 標準化}
    \begin{itemize}
        \item StandardScaler(平均0、標準偏差1)
        \item Trainデータで学習し、Val/Testに適用
    \end{itemize}
\end{frame}

%============================================
% セクション4: モデル実装
%============================================
\section{モデル実装}

%============================================
% スライド9: 使用モデル
%============================================
\begin{frame}{使用モデル}
    \begin{enumerate}
        \item \textbf{ロジスティック回帰}(PyTorch実装)
        \begin{itemize}
            \item 線形モデル、解釈性が高い
            \item ベースラインとして使用
        \end{itemize}

        \vspace{0.3cm}

        \item \textbf{MLP(Multi-Layer Perceptron)}(PyTorch実装)
        \begin{itemize}
            \item 非線形関係を学習可能
            \item BatchNorm + Dropoutで正則化
        \end{itemize}

        \vspace{0.3cm}

        \item \textbf{Random Forest}(scikit-learn)
        \begin{itemize}
            \item アンサンブル手法、過学習に強い
            \item 特徴量重要度を算出可能
        \end{itemize}
    \end{enumerate}
\end{frame}

%============================================
% スライド10: クラス不均衡対策
%============================================
\begin{frame}{クラス不均衡対策}
    \begin{block}{問題}
        Yes:No = 22:78 の不均衡データでは、「すべてNoと予測」でも78\%の精度
    \end{block}

    \vspace{0.3cm}

    \textbf{対策1: Weighted BCE Loss}
    \begin{equation}
        L = -\frac{1}{N}\sum_{i=1}^{N} [w_1 \cdot y_i \log(\hat{y}_i) + w_0 \cdot (1-y_i) \log(1-\hat{y}_i)]
    \end{equation}

    \textbf{対策2: Focal Loss}
    \begin{equation}
        L_{focal} = -\alpha (1-p_t)^\gamma \log(p_t)
    \end{equation}

    \textbf{対策3: class\_weight='balanced'}(Random Forest)
\end{frame}

%============================================
% セクション5: ハイパーパラメータチューニング
%============================================
\section{ハイパーパラメータチューニング}

%============================================
% スライド11: Optunaによる最適化
%============================================
\begin{frame}{Optunaによる最適化}
    \begin{columns}[T]
        \begin{column}{0.5\textwidth}
            \textbf{Optunaの特徴}
            \begin{itemize}
                \item TPE(Tree-structured Parzen Estimator)
                \item 自動枝刈り(Pruning)
                \item 効率的な探索空間の探索
            \end{itemize}

            \vspace{0.3cm}

            \textbf{設定}
            \begin{itemize}
                \item 各モデル: 50トライアル
                \item 最適化目標: Val AUC-ROC
                \item Pruner: MedianPruner
            \end{itemize}
        \end{column}

        \begin{column}{0.5\textwidth}
            \textbf{探索パラメータ}
            \begin{itemize}
                \item 学習率: $[10^{-5}, 10^{-1}]$
                \item 正則化: $[10^{-6}, 10^{-2}]$
                \item バッチサイズ: 64, 128, 256, 512
                \item 損失関数: Weighted BCE, Focal
                \item オプティマイザ: SGD, Adam, AdamW
            \end{itemize}
        \end{column}
    \end{columns}
\end{frame}

%============================================
% スライド12: ロジスティック回帰のチューニング
%============================================
\begin{frame}{ロジスティック回帰のチューニング}
    \textbf{探索パラメータ}
    \begin{itemize}
        \item 学習率: $[10^{-4}, 10^{-1}]$(対数スケール)
        \item 重み減衰(L2正則化): $[10^{-6}, 10^{-2}]$
        \item オプティマイザ: SGD, Adam
        \item 損失関数: Weighted BCE, Focal Loss
    \end{itemize}

    \vspace{0.5cm}

    \begin{figure}
        \centering
        \includegraphics[width=0.7\textwidth]{figures/learning_curves/logistic_regression.png}
        \caption{ロジスティック回帰の学習曲線}
    \end{figure}
\end{frame}

%============================================
% スライド13: MLPのチューニング
%============================================
\begin{frame}{MLPのチューニング}
    \textbf{探索パラメータ}
    \begin{itemize}
        \item 隠れ層数: 1〜4層
        \item 各層のユニット数: 32〜512
        \item Dropout率: 0.0〜0.5
        \item オプティマイザ: Adam, AdamW
    \end{itemize}

    \vspace{0.5cm}

    \begin{figure}
        \centering
        \includegraphics[width=0.7\textwidth]{figures/learning_curves/mlp.png}
        \caption{MLPの学習曲線}
    \end{figure}
\end{frame}

%============================================
% スライド14: Random Forestのチューニング
%============================================
\begin{frame}{Random Forestのチューニング}
    \textbf{探索パラメータ}
    \begin{itemize}
        \item n\_estimators: 50〜500
        \item max\_depth: 5〜50
        \item min\_samples\_split: 2〜20
        \item min\_samples\_leaf: 1〜10
        \item max\_features: 'sqrt', 'log2', None
    \end{itemize}

    \vspace{0.5cm}

    \begin{figure}
        \centering
        \includegraphics[width=0.6\textwidth]{figures/feature_importance/random_forest.png}
        \caption{Random Forestの特徴量重要度(上位10)}
    \end{figure}
\end{frame}

%============================================
% セクション6: 結果
%============================================
\section{実験結果}

%============================================
% スライド15: 評価指標
%============================================
\begin{frame}{評価指標}
    \begin{table}[h]
        \centering
        \begin{tabular}{lp{8cm}}
            \toprule
            指標 & 説明 \\
            \midrule
            Accuracy & 全体の正解率 \\
            Precision & 雨と予測した中で実際に雨だった割合 \\
            Recall & 実際の雨をどれだけ捕捉できたか \\
            F1-Score & PrecisionとRecallの調和平均 \\
            \textbf{AUC-ROC} & ROC曲線下面積(\textbf{主指標}) \\
            AUC-PR & Precision-Recall曲線下面積 \\
            \bottomrule
        \end{tabular}
    \end{table}

    \vspace{0.3cm}

    \begin{block}{AUC-ROCを主指標とする理由}
        閾値に依存しない評価が可能であり、クラス不均衡データに対して適切
    \end{block}
\end{frame}

%============================================
% スライド16: テストデータでの評価結果
%============================================
\begin{frame}{テストデータでの評価結果}
    \begin{table}[h]
        \centering
        \begin{tabular}{lcccccc}
            \toprule
            モデル & Accuracy & Precision & Recall & F1 & AUC-ROC & AUC-PR \\
            \midrule
            Logistic Reg. & 0.785 & 0.526 & 0.742 & 0.616 & 0.852 & 0.674 \\
            \textbf{MLP} & 0.797 & 0.542 & 0.785 & 0.641 & \textbf{0.875} & 0.720 \\
            Random Forest & 0.843 & 0.764 & 0.463 & 0.577 & 0.873 & 0.715 \\
            \bottomrule
        \end{tabular}
        \caption{各モデルの評価指標(テストデータ)}
    \end{table}

    \vspace{0.3cm}

    \begin{figure}
        \centering
        \includegraphics[width=0.7\textwidth]{figures/model_comparison.png}
        \caption{評価指標の比較}
    \end{figure}
\end{frame}

%============================================
% スライド17: ROCカーブ比較
%============================================
\begin{frame}{ROCカーブ比較}
    \begin{figure}
        \centering
        \includegraphics[width=0.65\textwidth]{figures/roc_curves/comparison.png}
        \caption{ROCカーブの比較}
    \end{figure}

    \begin{itemize}
        \item 曲線が左上に近いほど性能が良い
        \item 対角線はランダム予測(AUC=0.5)
    \end{itemize}
\end{frame}

%============================================
% スライド18: 混同行列
%============================================
\begin{frame}{混同行列}
    \begin{figure}
        \centering
        \includegraphics[width=0.9\textwidth]{figures/confusion_matrices/comparison.png}
        \caption{各モデルの混同行列}
    \end{figure}
\end{frame}

%============================================
% セクション7: 考察
%============================================
\section{考察}

%============================================
% スライド19: モデル比較の考察
%============================================
\begin{frame}{モデル比較の考察}
    \textbf{Random Forestが最高性能を達成}
    \begin{itemize}
        \item 非線形関係を効果的に捉える
        \item 特徴量間の相互作用を自動学習
        \item アンサンブル効果で過学習を抑制
    \end{itemize}

    \vspace{0.3cm}

    \textbf{MLPの特徴}
    \begin{itemize}
        \item 良好な性能だがチューニングに時間がかかる
        \item 適切なアーキテクチャ選択が重要
        \item GPU活用で学習を高速化
    \end{itemize}

    \vspace{0.3cm}

    \textbf{ロジスティック回帰の特徴}
    \begin{itemize}
        \item シンプルで解釈性が高い
        \item 非線形関係の捕捉に限界
        \item ベースラインとして有用
    \end{itemize}
\end{frame}

%============================================
% スライド20: 重要な特徴量
%============================================
\begin{frame}{重要な特徴量}
    Random Forestの特徴量重要度分析から:

    \vspace{0.3cm}

    \textbf{上位の特徴量}
    \begin{enumerate}
        \item \textbf{Humidity3pm}: 午後の湿度が最重要
        \item \textbf{Pressure3pm}: 気圧も重要な予測因子
        \item \textbf{Sunshine}: 日照時間
        \item \textbf{Cloud3pm}: 午後の雲量
        \item \textbf{RainToday}: 本日の雨の有無
    \end{enumerate}

    \vspace{0.3cm}

    \begin{block}{気象学的解釈}
        午後の湿度・気圧・雲量は、翌日の降雨を予測する上で物理的に妥当な指標
    \end{block}
\end{frame}

%============================================
% セクション8: まとめ
%============================================
\section{まとめ}

%============================================
% スライド21: まとめ
%============================================
\begin{frame}{まとめ}
    \textbf{本研究の成果}
    \begin{enumerate}
        \item 3種類のモデル(ロジスティック回帰、MLP、Random Forest)を実装・比較
        \item Optunaによる体系的なハイパーパラメータチューニングを実施
        \item クラス不均衡に対する適切な対策(Weighted Loss, class\_weight)を適用
        \item Random Forestが最高のAUC-ROCを達成
    \end{enumerate}

    \vspace{0.5cm}

    \textbf{今後の課題}
    \begin{itemize}
        \item 勾配ブースティング(LightGBM、XGBoost)の追加
        \item 時系列特徴量(ラグ特徴量、移動平均)の導入
        \item モデルアンサンブル(スタッキング)の検討
        \item より多くのトライアルでのハイパーパラメータ探索
    \end{itemize}
\end{frame}

%============================================
% スライド22: 参考文献
%============================================
\begin{frame}{参考文献}
    \begin{thebibliography}{9}
        \bibitem{kaggle}
        Young, J. (2017). Rain in Australia. Kaggle Dataset.
        \url{https://www.kaggle.com/datasets/jsphyg/weather-dataset-rattle-package}

        \bibitem{pytorch}
        Paszke, A., et al. (2019). PyTorch: An Imperative Style, High-Performance Deep Learning Library. NeurIPS.

        \bibitem{optuna}
        Akiba, T., et al. (2019). Optuna: A Next-generation Hyperparameter Optimization Framework. KDD.

        \bibitem{sklearn}
        Pedregosa, F., et al. (2011). Scikit-learn: Machine Learning in Python. JMLR.
    \end{thebibliography}
\end{frame}

\end{document}
